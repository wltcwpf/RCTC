\documentclass[]{article}
\usepackage{lmodern}
\usepackage{amssymb,amsmath}
\usepackage{ifxetex,ifluatex}
\usepackage{fixltx2e} % provides \textsubscript
\ifnum 0\ifxetex 1\fi\ifluatex 1\fi=0 % if pdftex
  \usepackage[T1]{fontenc}
  \usepackage[utf8]{inputenc}
\else % if luatex or xelatex
  \ifxetex
    \usepackage{mathspec}
  \else
    \usepackage{fontspec}
  \fi
  \defaultfontfeatures{Ligatures=TeX,Scale=MatchLowercase}
\fi
% use upquote if available, for straight quotes in verbatim environments
\IfFileExists{upquote.sty}{\usepackage{upquote}}{}
% use microtype if available
\IfFileExists{microtype.sty}{%
\usepackage{microtype}
\UseMicrotypeSet[protrusion]{basicmath} % disable protrusion for tt fonts
}{}
\usepackage[margin=1in]{geometry}
\usepackage{hyperref}
\hypersetup{unicode=true,
            pdftitle={GM\_RotD.R},
            pdfauthor={PFW},
            pdfborder={0 0 0},
            breaklinks=true}
\urlstyle{same}  % don't use monospace font for urls
\usepackage{color}
\usepackage{fancyvrb}
\newcommand{\VerbBar}{|}
\newcommand{\VERB}{\Verb[commandchars=\\\{\}]}
\DefineVerbatimEnvironment{Highlighting}{Verbatim}{commandchars=\\\{\}}
% Add ',fontsize=\small' for more characters per line
\usepackage{framed}
\definecolor{shadecolor}{RGB}{248,248,248}
\newenvironment{Shaded}{\begin{snugshade}}{\end{snugshade}}
\newcommand{\KeywordTok}[1]{\textcolor[rgb]{0.13,0.29,0.53}{\textbf{#1}}}
\newcommand{\DataTypeTok}[1]{\textcolor[rgb]{0.13,0.29,0.53}{#1}}
\newcommand{\DecValTok}[1]{\textcolor[rgb]{0.00,0.00,0.81}{#1}}
\newcommand{\BaseNTok}[1]{\textcolor[rgb]{0.00,0.00,0.81}{#1}}
\newcommand{\FloatTok}[1]{\textcolor[rgb]{0.00,0.00,0.81}{#1}}
\newcommand{\ConstantTok}[1]{\textcolor[rgb]{0.00,0.00,0.00}{#1}}
\newcommand{\CharTok}[1]{\textcolor[rgb]{0.31,0.60,0.02}{#1}}
\newcommand{\SpecialCharTok}[1]{\textcolor[rgb]{0.00,0.00,0.00}{#1}}
\newcommand{\StringTok}[1]{\textcolor[rgb]{0.31,0.60,0.02}{#1}}
\newcommand{\VerbatimStringTok}[1]{\textcolor[rgb]{0.31,0.60,0.02}{#1}}
\newcommand{\SpecialStringTok}[1]{\textcolor[rgb]{0.31,0.60,0.02}{#1}}
\newcommand{\ImportTok}[1]{#1}
\newcommand{\CommentTok}[1]{\textcolor[rgb]{0.56,0.35,0.01}{\textit{#1}}}
\newcommand{\DocumentationTok}[1]{\textcolor[rgb]{0.56,0.35,0.01}{\textbf{\textit{#1}}}}
\newcommand{\AnnotationTok}[1]{\textcolor[rgb]{0.56,0.35,0.01}{\textbf{\textit{#1}}}}
\newcommand{\CommentVarTok}[1]{\textcolor[rgb]{0.56,0.35,0.01}{\textbf{\textit{#1}}}}
\newcommand{\OtherTok}[1]{\textcolor[rgb]{0.56,0.35,0.01}{#1}}
\newcommand{\FunctionTok}[1]{\textcolor[rgb]{0.00,0.00,0.00}{#1}}
\newcommand{\VariableTok}[1]{\textcolor[rgb]{0.00,0.00,0.00}{#1}}
\newcommand{\ControlFlowTok}[1]{\textcolor[rgb]{0.13,0.29,0.53}{\textbf{#1}}}
\newcommand{\OperatorTok}[1]{\textcolor[rgb]{0.81,0.36,0.00}{\textbf{#1}}}
\newcommand{\BuiltInTok}[1]{#1}
\newcommand{\ExtensionTok}[1]{#1}
\newcommand{\PreprocessorTok}[1]{\textcolor[rgb]{0.56,0.35,0.01}{\textit{#1}}}
\newcommand{\AttributeTok}[1]{\textcolor[rgb]{0.77,0.63,0.00}{#1}}
\newcommand{\RegionMarkerTok}[1]{#1}
\newcommand{\InformationTok}[1]{\textcolor[rgb]{0.56,0.35,0.01}{\textbf{\textit{#1}}}}
\newcommand{\WarningTok}[1]{\textcolor[rgb]{0.56,0.35,0.01}{\textbf{\textit{#1}}}}
\newcommand{\AlertTok}[1]{\textcolor[rgb]{0.94,0.16,0.16}{#1}}
\newcommand{\ErrorTok}[1]{\textcolor[rgb]{0.64,0.00,0.00}{\textbf{#1}}}
\newcommand{\NormalTok}[1]{#1}
\usepackage{graphicx,grffile}
\makeatletter
\def\maxwidth{\ifdim\Gin@nat@width>\linewidth\linewidth\else\Gin@nat@width\fi}
\def\maxheight{\ifdim\Gin@nat@height>\textheight\textheight\else\Gin@nat@height\fi}
\makeatother
% Scale images if necessary, so that they will not overflow the page
% margins by default, and it is still possible to overwrite the defaults
% using explicit options in \includegraphics[width, height, ...]{}
\setkeys{Gin}{width=\maxwidth,height=\maxheight,keepaspectratio}
\IfFileExists{parskip.sty}{%
\usepackage{parskip}
}{% else
\setlength{\parindent}{0pt}
\setlength{\parskip}{6pt plus 2pt minus 1pt}
}
\setlength{\emergencystretch}{3em}  % prevent overfull lines
\providecommand{\tightlist}{%
  \setlength{\itemsep}{0pt}\setlength{\parskip}{0pt}}
\setcounter{secnumdepth}{0}
% Redefines (sub)paragraphs to behave more like sections
\ifx\paragraph\undefined\else
\let\oldparagraph\paragraph
\renewcommand{\paragraph}[1]{\oldparagraph{#1}\mbox{}}
\fi
\ifx\subparagraph\undefined\else
\let\oldsubparagraph\subparagraph
\renewcommand{\subparagraph}[1]{\oldsubparagraph{#1}\mbox{}}
\fi

%%% Use protect on footnotes to avoid problems with footnotes in titles
\let\rmarkdownfootnote\footnote%
\def\footnote{\protect\rmarkdownfootnote}

%%% Change title format to be more compact
\usepackage{titling}

% Create subtitle command for use in maketitle
\providecommand{\subtitle}[1]{
  \posttitle{
    \begin{center}\large#1\end{center}
    }
}

\setlength{\droptitle}{-2em}

  \title{GM\_RotD.R}
    \pretitle{\vspace{\droptitle}\centering\huge}
  \posttitle{\par}
    \author{PFW}
    \preauthor{\centering\large\emph}
  \postauthor{\par}
      \predate{\centering\large\emph}
  \postdate{\par}
    \date{2020-02-21}


\begin{document}
\maketitle

A Function for Combination of Two Rotated Ground Motions

This function computes RotD with 180 rotated angles and Geometric Mean
of RotD with 90 angles @param data1 The first horizontal component of
ground motion @param data2 The second horizontal component of ground
motion @param period\_t A array of oscillator periods @param damping
Damping ratio @param time\_dt Time step of two time series. These two
datasets have the same time step @param fraction A fraction for subset
selection to speed up calculation. 0 is using all dataset. 0.7 is
recommended for RotD50 if the input data are Sinc interpolated
(default); 0.5 is recommended if GMRotI50 is also interested; 0.0 should
be used regarding for RotD00 @param Interpolation\_factor The specific
value depends on sampling rate, the detailed explanation is described in
the PEER report. Users can give their desired value, or can use defult
``auto'', which will compute \code{Interpolation_factor} according to
time step automatically. @return A list is returned with RotD, GMRotD,
number of selected subsets points, threshold values, PGARotD, PGVRotD,
and PGDRotD @keywords GM\_RotD\_cal @export @examples
GM\_RotD\_cal(c(0.01,0.03,\ldots{}0.5), c(-0.01,0.02,\ldots{}0.05),
c(0.01,0.02,\ldots{}10), 0.05, 0.005, 0.7)

\begin{Shaded}
\begin{Highlighting}[]
\NormalTok{########################################}
\NormalTok{######## inputs: time series, interpolation multiplier}
\NormalTok{######## outputs: Sinc interpolated time series}


\NormalTok{GM_RotD_cal <-}\StringTok{ }\ControlFlowTok{function}\NormalTok{(data1, data2, period_t, damping, time_dt, }\DataTypeTok{fraction =} \FloatTok{0.7}\NormalTok{, }\DataTypeTok{Interpolation_factor =} \StringTok{"auto"}\NormalTok{)\{}
\NormalTok{  ## Sinc interpolation}
  \ControlFlowTok{if}\NormalTok{(Interpolation_factor }\OperatorTok{==}\StringTok{ 'auto'}\NormalTok{)\{}
    \CommentTok{# interp_factor <- ceiling(log(time_dt/0.001)/log(2))}

    \CommentTok{# 1) the interpolation factor is based on f_nyq first -- we only care T_nyq,}
    \CommentTok{# because T less than T_nyq is not affected by interpolation,}
    \CommentTok{# therefore, dt/dt' = dt/(T_nyq/10) = dt/(dt/5) = 5, -> IF = 8}
    \CommentTok{# 2) if time_dt is less than the above calcualtion, then we use smaller interpolation}
\NormalTok{    interp_factor <-}\StringTok{ }\KeywordTok{min}\NormalTok{(}\KeywordTok{ceiling}\NormalTok{(}\KeywordTok{log}\NormalTok{(time_dt}\OperatorTok{/}\FloatTok{0.001}\NormalTok{)}\OperatorTok{/}\KeywordTok{log}\NormalTok{(}\DecValTok{2}\NormalTok{)), }\DecValTok{3}\NormalTok{)}
    \CommentTok{# if time_dt even smaller than the diresed time step (not common), we do not interpolate}
\NormalTok{    interp_factor <-}\StringTok{ }\KeywordTok{max}\NormalTok{(interp_factor, }\DecValTok{0}\NormalTok{)}
\NormalTok{    interp_factor <-}\StringTok{ }\DecValTok{2}\OperatorTok{^}\NormalTok{interp_factor}
\NormalTok{  \}}\ControlFlowTok{else} \ControlFlowTok{if}\NormalTok{(}\KeywordTok{log}\NormalTok{(Interpolation_factor)}\OperatorTok\KeywordTok{log}\NormalTok{(}\DecValTok{2}\NormalTok{) }\OperatorTok{!=}\StringTok{ }\DecValTok{0}\NormalTok{)\{}
    \KeywordTok{print}\NormalTok{(}\StringTok{"The Interpolation factor is not a number of power of 2, please try correct it!"}\NormalTok{)}
    \KeywordTok{stop}\NormalTok{()}
\NormalTok{  \}}\ControlFlowTok{else}\NormalTok{\{}
\NormalTok{    interp_factor <-}\StringTok{ }\NormalTok{Interpolation_factor}
\NormalTok{  \}}
\NormalTok{  time_dt <-}\StringTok{ }\NormalTok{time_dt}\OperatorTok{/}\NormalTok{interp_factor }\CommentTok{# compute new time step}
\NormalTok{  data1 <-}\StringTok{ }\KeywordTok{Interpft}\NormalTok{(data1, interp_factor) }\CommentTok{# new data after Sinc interpolation}
\NormalTok{  data2 <-}\StringTok{ }\KeywordTok{Interpft}\NormalTok{(data2, interp_factor)}
\NormalTok{  number <-}\StringTok{ }\KeywordTok{min}\NormalTok{(}\KeywordTok{length}\NormalTok{(data1), }\KeywordTok{length}\NormalTok{(data2)) }\CommentTok{# if two data sizes diff, we set both size as the smaller}
\NormalTok{  data1 <-}\StringTok{ }\NormalTok{data1[}\DecValTok{1}\OperatorTok{:}\NormalTok{number]}
\NormalTok{  data2 <-}\StringTok{ }\NormalTok{data2[}\DecValTok{1}\OperatorTok{:}\NormalTok{number]}
\NormalTok{  vel_}\DecValTok{1}\NormalTok{ <-}\StringTok{ }\KeywordTok{cumtrapz}\NormalTok{(time_dt}\OperatorTok{*}\KeywordTok{seq}\NormalTok{(}\DecValTok{1}\NormalTok{,}\KeywordTok{length}\NormalTok{(data1)),data1) }\OperatorTok{*}\StringTok{ }\DecValTok{981} \CommentTok{# unit, cm/s}
\NormalTok{  vel_}\DecValTok{2}\NormalTok{ <-}\StringTok{ }\KeywordTok{cumtrapz}\NormalTok{(time_dt}\OperatorTok{*}\KeywordTok{seq}\NormalTok{(}\DecValTok{1}\NormalTok{,}\KeywordTok{length}\NormalTok{(data2)),data2) }\OperatorTok{*}\StringTok{ }\DecValTok{981} \CommentTok{# unit, cm/s}
\NormalTok{  disp_}\DecValTok{1}\NormalTok{ <-}\StringTok{ }\KeywordTok{cumtrapz}\NormalTok{(time_dt}\OperatorTok{*}\KeywordTok{seq}\NormalTok{(}\DecValTok{1}\NormalTok{,}\KeywordTok{length}\NormalTok{(vel_}\DecValTok{1}\NormalTok{)),vel_}\DecValTok{1}\NormalTok{) }\CommentTok{# unit, cm}
\NormalTok{  disp_}\DecValTok{2}\NormalTok{ <-}\StringTok{ }\KeywordTok{cumtrapz}\NormalTok{(time_dt}\OperatorTok{*}\KeywordTok{seq}\NormalTok{(}\DecValTok{1}\NormalTok{,}\KeywordTok{length}\NormalTok{(vel_}\DecValTok{2}\NormalTok{)),vel_}\DecValTok{2}\NormalTok{) }\CommentTok{# unit, cm}
\NormalTok{  ### subset slection}
\NormalTok{  length_min <-}\StringTok{ }\KeywordTok{min}\NormalTok{(}\KeywordTok{length}\NormalTok{(data1), }\KeywordTok{length}\NormalTok{(data2))}
\NormalTok{  a_subset <-}\StringTok{ }\KeywordTok{subset_select}\NormalTok{(data1, data2, fraction, length_min, time_dt, }\DecValTok{1}\NormalTok{)}
\NormalTok{  v_subset <-}\StringTok{ }\KeywordTok{subset_select}\NormalTok{(vel_}\DecValTok{1}\NormalTok{, vel_}\DecValTok{2}\NormalTok{, fraction, length_min, time_dt, }\DecValTok{1}\NormalTok{)}
\NormalTok{  d_subset <-}\StringTok{ }\KeywordTok{subset_select}\NormalTok{(disp_}\DecValTok{1}\NormalTok{, disp_}\DecValTok{2}\NormalTok{, fraction, length_min, time_dt, }\DecValTok{1}\NormalTok{)}
\NormalTok{  ### Compute for RotD of PGA, PGV, PGD}
\NormalTok{  pga_rot <-}\StringTok{ }\KeywordTok{c}\NormalTok{()}
\NormalTok{  pgv_rot <-}\StringTok{ }\KeywordTok{c}\NormalTok{()}
\NormalTok{  pgd_rot <-}\StringTok{ }\KeywordTok{c}\NormalTok{()}
  \ControlFlowTok{for}\NormalTok{(theta }\ControlFlowTok{in} \KeywordTok{seq}\NormalTok{(}\DecValTok{1}\NormalTok{,}\DecValTok{90}\NormalTok{))\{}
    \CommentTok{# acceleration}
\NormalTok{    Rot_a1 <-}\StringTok{ }\NormalTok{a_subset[}\DecValTok{1}\NormalTok{,]}\OperatorTok{*}\KeywordTok{cos}\NormalTok{(theta}\OperatorTok{/}\DecValTok{180}\OperatorTok{*}\NormalTok{pi) }\OperatorTok{+}\StringTok{ }\NormalTok{a_subset[}\DecValTok{2}\NormalTok{,]}\OperatorTok{*}\KeywordTok{sin}\NormalTok{(theta}\OperatorTok{/}\DecValTok{180}\OperatorTok{*}\NormalTok{pi)}
\NormalTok{    Rot_a2 <-}\StringTok{ }\OperatorTok{-}\NormalTok{a_subset[}\DecValTok{1}\NormalTok{,]}\OperatorTok{*}\KeywordTok{sin}\NormalTok{(theta}\OperatorTok{/}\DecValTok{180}\OperatorTok{*}\NormalTok{pi) }\OperatorTok{+}\StringTok{ }\NormalTok{a_subset[}\DecValTok{2}\NormalTok{,]}\OperatorTok{*}\KeywordTok{cos}\NormalTok{(theta}\OperatorTok{/}\DecValTok{180}\OperatorTok{*}\NormalTok{pi)}
\NormalTok{    pga_rot[theta] <-}\StringTok{ }\KeywordTok{max}\NormalTok{(}\KeywordTok{abs}\NormalTok{(Rot_a1))}
\NormalTok{    pga_rot[theta}\OperatorTok{+}\DecValTok{90}\NormalTok{] <-}\StringTok{ }\KeywordTok{max}\NormalTok{(}\KeywordTok{abs}\NormalTok{(Rot_a2))}
    \CommentTok{# velocity}
\NormalTok{    Rot_v1 <-}\StringTok{ }\NormalTok{v_subset[}\DecValTok{1}\NormalTok{,]}\OperatorTok{*}\KeywordTok{cos}\NormalTok{(theta}\OperatorTok{/}\DecValTok{180}\OperatorTok{*}\NormalTok{pi) }\OperatorTok{+}\StringTok{ }\NormalTok{v_subset[}\DecValTok{2}\NormalTok{,]}\OperatorTok{*}\KeywordTok{sin}\NormalTok{(theta}\OperatorTok{/}\DecValTok{180}\OperatorTok{*}\NormalTok{pi)}
\NormalTok{    Rot_v2 <-}\StringTok{ }\OperatorTok{-}\NormalTok{v_subset[}\DecValTok{1}\NormalTok{,]}\OperatorTok{*}\KeywordTok{sin}\NormalTok{(theta}\OperatorTok{/}\DecValTok{180}\OperatorTok{*}\NormalTok{pi) }\OperatorTok{+}\StringTok{ }\NormalTok{v_subset[}\DecValTok{2}\NormalTok{,]}\OperatorTok{*}\KeywordTok{cos}\NormalTok{(theta}\OperatorTok{/}\DecValTok{180}\OperatorTok{*}\NormalTok{pi)}
\NormalTok{    pgv_rot[theta] <-}\StringTok{ }\KeywordTok{max}\NormalTok{(}\KeywordTok{abs}\NormalTok{(Rot_v1))}
\NormalTok{    pgv_rot[theta}\OperatorTok{+}\DecValTok{90}\NormalTok{] <-}\StringTok{ }\KeywordTok{max}\NormalTok{(}\KeywordTok{abs}\NormalTok{(Rot_v2))}
    \CommentTok{# displacement}
\NormalTok{    Rot_d1 <-}\StringTok{ }\NormalTok{d_subset[}\DecValTok{1}\NormalTok{,]}\OperatorTok{*}\KeywordTok{cos}\NormalTok{(theta}\OperatorTok{/}\DecValTok{180}\OperatorTok{*}\NormalTok{pi) }\OperatorTok{+}\StringTok{ }\NormalTok{d_subset[}\DecValTok{2}\NormalTok{,]}\OperatorTok{*}\KeywordTok{sin}\NormalTok{(theta}\OperatorTok{/}\DecValTok{180}\OperatorTok{*}\NormalTok{pi)}
\NormalTok{    Rot_d2 <-}\StringTok{ }\OperatorTok{-}\NormalTok{d_subset[}\DecValTok{1}\NormalTok{,]}\OperatorTok{*}\KeywordTok{sin}\NormalTok{(theta}\OperatorTok{/}\DecValTok{180}\OperatorTok{*}\NormalTok{pi) }\OperatorTok{+}\StringTok{ }\NormalTok{d_subset[}\DecValTok{2}\NormalTok{,]}\OperatorTok{*}\KeywordTok{cos}\NormalTok{(theta}\OperatorTok{/}\DecValTok{180}\OperatorTok{*}\NormalTok{pi)}
\NormalTok{    pgd_rot[theta] <-}\StringTok{ }\KeywordTok{max}\NormalTok{(}\KeywordTok{abs}\NormalTok{(Rot_d1))}
\NormalTok{    pgd_rot[theta}\OperatorTok{+}\DecValTok{90}\NormalTok{] <-}\StringTok{ }\KeywordTok{max}\NormalTok{(}\KeywordTok{abs}\NormalTok{(Rot_d2))}
\NormalTok{  \}}

  \CommentTok{# row is period sequence for a given rotated angle, colomn is angle sequence for a given period}
\NormalTok{  GMRotD <-}\StringTok{ }\KeywordTok{matrix}\NormalTok{(}\DataTypeTok{nrow =} \KeywordTok{length}\NormalTok{(}\KeywordTok{seq}\NormalTok{(}\DecValTok{1}\NormalTok{,}\DecValTok{90}\NormalTok{)), }\DataTypeTok{ncol =} \KeywordTok{length}\NormalTok{(period_t))}
\NormalTok{  RotD180 <-}\StringTok{ }\KeywordTok{matrix}\NormalTok{(}\DataTypeTok{nrow =} \KeywordTok{length}\NormalTok{(}\KeywordTok{seq}\NormalTok{(}\DecValTok{1}\NormalTok{,}\DecValTok{180}\NormalTok{)), }\DataTypeTok{ncol =} \KeywordTok{length}\NormalTok{(period_t))}
  \CommentTok{# record the number of selected points}
\NormalTok{  len_period <-}\StringTok{ }\KeywordTok{length}\NormalTok{(period_t)}
\NormalTok{  Num_points <-}\StringTok{ }\KeywordTok{seq}\NormalTok{(}\DecValTok{1}\NormalTok{,len_period)}
\NormalTok{  rd_alevel <-}\StringTok{ }\KeywordTok{seq}\NormalTok{(}\DecValTok{1}\NormalTok{,len_period)}
\NormalTok{  RD1 <-}\StringTok{ }\KeywordTok{PS_cal_cpp}\NormalTok{(data1, period_t, damping, time_dt, }\DecValTok{2}\NormalTok{) }\CommentTok{# matrix [period by data points]}
\NormalTok{  RD2 <-}\StringTok{ }\KeywordTok{PS_cal_cpp}\NormalTok{(data2, period_t, damping, time_dt, }\DecValTok{2}\NormalTok{) }\CommentTok{# matrix [period by data points]}
\NormalTok{  omega =}\StringTok{ }\DecValTok{2}\OperatorTok{*}\NormalTok{pi}\OperatorTok{/}\NormalTok{period_t}
  \ControlFlowTok{for}\NormalTok{(per_index }\ControlFlowTok{in} \KeywordTok{seq}\NormalTok{(}\DecValTok{1}\NormalTok{,len_period))\{}
\NormalTok{    length_min <-}\StringTok{ }\KeywordTok{min}\NormalTok{(}\KeywordTok{length}\NormalTok{(RD1[per_index,]), }\KeywordTok{length}\NormalTok{(RD2[per_index,]))}
\NormalTok{    rd_subset <-}\StringTok{ }\KeywordTok{subset_select}\NormalTok{(RD1[per_index,], RD2[per_index,], fraction, length_min, time_dt, }\DecValTok{1}\NormalTok{)}
\NormalTok{    rd_alevel[per_index] <-}\StringTok{ }\KeywordTok{subset_select}\NormalTok{(RD1[per_index,], RD2[per_index,], fraction, length_min, time_dt, }\DecValTok{2}\NormalTok{)}
\NormalTok{    Num_points[per_index] <-}\StringTok{ }\KeywordTok{length}\NormalTok{(rd_subset)}\OperatorTok{/}\DecValTok{3}
\NormalTok{    Rot_rd1 <-}\StringTok{ }\NormalTok{rd_subset[}\DecValTok{1}\NormalTok{,]}
\NormalTok{    Rot_rd2 <-}\StringTok{ }\NormalTok{rd_subset[}\DecValTok{2}\NormalTok{,]}
\NormalTok{    rot_angle4rot <-}\StringTok{ }\KeywordTok{seq}\NormalTok{(}\DecValTok{1}\NormalTok{,}\DecValTok{180}\NormalTok{)}
    \ControlFlowTok{for}\NormalTok{(theta }\ControlFlowTok{in} \KeywordTok{seq}\NormalTok{(}\DecValTok{1}\NormalTok{,}\DecValTok{90}\NormalTok{))\{}
\NormalTok{      RS1 =}\StringTok{ }\NormalTok{Rot_rd1}\OperatorTok{*}\KeywordTok{cos}\NormalTok{(theta}\OperatorTok{/}\DecValTok{180}\OperatorTok{*}\NormalTok{pi) }\OperatorTok{+}\StringTok{ }\NormalTok{Rot_rd2}\OperatorTok{*}\KeywordTok{sin}\NormalTok{(theta}\OperatorTok{/}\DecValTok{180}\OperatorTok{*}\NormalTok{pi)}
\NormalTok{      RS2 =}\StringTok{ }\OperatorTok{-}\NormalTok{Rot_rd1}\OperatorTok{*}\KeywordTok{sin}\NormalTok{(theta}\OperatorTok{/}\DecValTok{180}\OperatorTok{*}\NormalTok{pi) }\OperatorTok{+}\StringTok{ }\NormalTok{Rot_rd2}\OperatorTok{*}\KeywordTok{cos}\NormalTok{(theta}\OperatorTok{/}\DecValTok{180}\OperatorTok{*}\NormalTok{pi)}
\NormalTok{      RotD180[theta, per_index] =}\StringTok{ }\KeywordTok{max}\NormalTok{(}\KeywordTok{abs}\NormalTok{(RS1))}\OperatorTok{*}\NormalTok{omega[per_index]}\OperatorTok{^}\DecValTok{2}
\NormalTok{      RotD180[theta}\OperatorTok{+}\DecValTok{90}\NormalTok{, per_index] =}\StringTok{ }\KeywordTok{max}\NormalTok{(}\KeywordTok{abs}\NormalTok{(RS2))}\OperatorTok{*}\NormalTok{omega[per_index]}\OperatorTok{^}\DecValTok{2}
\NormalTok{      GMRotD[theta,per_index] =}\StringTok{ }\KeywordTok{sqrt}\NormalTok{(}\KeywordTok{max}\NormalTok{(}\KeywordTok{abs}\NormalTok{(RS1))}\OperatorTok{*}\KeywordTok{max}\NormalTok{(}\KeywordTok{abs}\NormalTok{(RS2)))}\OperatorTok{*}\NormalTok{omega[per_index]}\OperatorTok{^}\DecValTok{2}
\NormalTok{    \}}
\NormalTok{  \}}

\NormalTok{  gm <-}\StringTok{ }\KeywordTok{list}\NormalTok{()}
\NormalTok{  gm}\OperatorTok{$}\NormalTok{RotD180 <-}\StringTok{ }\NormalTok{RotD180}
\NormalTok{  gm}\OperatorTok{$}\NormalTok{GMRotD <-}\StringTok{ }\NormalTok{GMRotD}
\NormalTok{  gm}\OperatorTok{$}\NormalTok{Num_points <-}\StringTok{ }\NormalTok{Num_points}
\NormalTok{  gm}\OperatorTok{$}\NormalTok{rd_level <-}\StringTok{ }\NormalTok{rd_alevel}
\NormalTok{  gm}\OperatorTok{$}\NormalTok{pga_rot <-}\StringTok{ }\NormalTok{pga_rot}
\NormalTok{  gm}\OperatorTok{$}\NormalTok{pgv_rot <-}\StringTok{ }\NormalTok{pgv_rot}
\NormalTok{  gm}\OperatorTok{$}\NormalTok{pgd_rot <-}\StringTok{ }\NormalTok{pgd_rot}
  \KeywordTok{return}\NormalTok{(gm)}
\NormalTok{\}}
\end{Highlighting}
\end{Shaded}


\end{document}
